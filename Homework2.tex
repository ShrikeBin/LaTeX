\documentclass{article}
\usepackage{amsmath}
\usepackage{amssymb}
\usepackage{graphicx}
\usepackage{listings}
\usepackage{xcolor}
\usepackage{geometry}
\usepackage[utf8]{inputenc}
\usepackage[T1]{fontenc}
\usepackage[polish]{babel}
\geometry{a4paper, margin=1in}

\title{Raport z Symulacji Urn}
\author{Jan Ryszkiewicz}
\date{}

\begin{document}

\maketitle

\section*{Opis wyników}

W ramach zadania przeprowadziłem symulację wrzucana kul do n urn po 50 razy śledząc następujące informacje:
\begin{itemize}
    \item \( B \) moment pierwszej kolizji,
    \item \( C \) minimalna liczba rzutów, po której w kazdej z urn jest co najmniej jedna kula,
    \item \( D \) minimalna liczba rzutów, po której w kazdej z urn są co najmniej dwie kule,
    \item \( U \) liczba pustych urn po wrzuceniu n kul.
    \item \( D - C \) liczba rzutów od momentu Cn potrzebna do tego, zeby w każdej urnie były conajmniej dwie kule.
\end{itemize}

\noindent Na wykresach (pliki .png w folderze /plots) przedstawiam wyniki pojedynczych symulacji oraz ich średnie wartości dla danych n.

\section*{Wnioski na temat wartości śledzonych danych}

\begin{itemize}
    \item \( B \) warto zauważyć, że pierwsza kolizja ma mniejsce bardzo szybko
    \item \( C \) bardzo duże wartości w stosunku do n, odchylenie rośnie wraz z n
    \item \( D \) podobnie do C(n)
    \item \( U \) rośnie stale liniowo, odhylenie od średniej jest minimalne
    \item \( D - C \) bardzo chaotyczne wyniki, średnia wizualizuje ogólną tendencję

\end{itemize}

\section*{Wnioski na temat asymptotyki}

\begin{itemize}
    \item \textbf{} \(\frac{b(n)}{n}\)
    - zbiega do 0
    \item \textbf{} \(\frac{b(n)}{\sqrt{n}}\)
    - wydaje się oscylować wokół 1 czyli te funkcje uzyskują podobne wartości
    \item \textbf{} \(\frac{u(n)}{n}\)
    - między 0.365 a 0.370, tak samo jak poprzednia, wydaje się stałą więc asymptotycznie funkcje są równe
    \item \textbf{} \(\frac{c(n)}{n}\)
    - wydaje się rosnąć, kształt przypomina \(\sqrt{x}\), możliwe że c(n) jest podobna do n\(\sqrt{n}\)
    \item \textbf{} \(\frac{c(n)}{n^2}\)
    - zbiega do 0
    \item \textbf{} \(\frac{c(n)}{n \ln{(n)}}\)
    - oscyluje nieco powyżej 1 co sugeruje asymptotyczną równość
    \item \textbf{} \(\frac{d(n)}{n}\)
    - wydaje się rosnąć, kształt przypomina \(\sqrt{x}\), możliwe że d(n) jest podobna do n\(\sqrt{n}\)
    \item \textbf{} \(\frac{d(n)}{n^2}\)
    - zbiega do 0
    \item \textbf{} \(\frac{d(n)}{n \ln{(n)}}\)
    - wydaje się bardzo powolnie zbiegać do 0, ciężko określić
    \item \textbf{} \(\frac{d(n) - c(n)}{n}\)
    - oscyluje nieco powyżej 2 wydaje się stałą więc prawdopodobnie asymptotycznie funkcje są równe
    \item \textbf{} \(\frac{d(n) - c(n)}{n \ln{(n)}}\)
    - wydaje się bardzo powolnie zbiegać do 0, ciężko określić
    \item \textbf{} \(\frac{d(n) - c(n)}{n \ln{\ln{(n)}}}\)
    - funkcje są sobie równe, iloraz jest asymptotycznie = 1
\end{itemize}

\section*{\textsl{Birthday paradox} oraz \textsl{coupon collector's problem}}
\subsection{Użycie nazw}
\begin{itemize}
    \item \textsl{Paradoks urodzinowy} - jak szybko dojdziemy do kolizji (ten sam dzień urodzin) w pewnej grupie ludzi jest dobrym obrazem prezentowanych danych,
    szansa na kolizję rośnie znacznie szybciej niż mogłoby się wydawać stąd nazwa "paradoks"
    \item \textsl{Problem zbieracza kuponów} - załóżmy że mamy do zebrania 10 kuponów i w każdej paczce jest losowy kupon, jeżeli chcemy zebrać wszystkie kupony to musimy
    średnio otworzyć własnie taką liczbę paczek, stąd nazwa.
\end{itemize}
\subsection{\textsl{Birthday paradox} a funkcje hashujące}
"Birthday paradox" pokazuje jak łatwo może dojść do kolizji w funkcji hashującej. 
Sam paradoks mówi że jeśli w pokoju znajdują się 23 osoby to szansa na to że mają urodziny tego
samego dnia wynosi >50\% co pokazuje jak szybko wbrew intuicji szansa na kolizję (tutaj daty urodzin) wzrasta.

\hfill
\textbf{Jan Ryszkiewicz}

\end{document}
