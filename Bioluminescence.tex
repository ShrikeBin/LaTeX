\documentclass{article}
\usepackage{amsmath}
\usepackage{amssymb}
\usepackage{graphicx}
\usepackage{listings}
\usepackage{xcolor}
\usepackage{geometry}
\geometry{a4paper, margin=1in}

\title{Bioluminescence}
\author{Nel Skowronek, Jan Ryszkiewicz}
\date{}

\begin{document}

\maketitle

\section*{Bioluminescence}
Bioluminescence is the ability of living organisms to produce light through chemical reactions. \\
It occurs in various organisms, including marine creatures, fungi, and insects.\\
The primary mechanism involves the interaction of a molecule called \textbf{luciferin} with an enzyme called \textbf{luciferase}.


\section*{Fireflies (6)}

\subsection*{- Intro }
Fireflies, also known as lightning bugs, are famous for their ability to produce bioluminescence.\\ 
This light is a result of a chemical reaction within their lower abdomen.\\ 
Fireflies use their light primarily for \textbf{mating signals}.

\subsection*{- What is Luciferin? }
Luciferin is a small molecule that produces light when it oxidizes through \textbf{luciferase}
There is a lot of different types of luciferin and corresponding luciferase, studying them \\
makes it possible to determine how closely related two species are.\\
Carbon bounds.....

\subsection*{- Luciferin in Fireflies }
In fireflies, when they shake thier butt and mix the two compounds:\\
luciferin combines with luciferase, ATP, and oxygen to produce light.\\
The efficiency of light emission in fireflies is extremely high.\\

\subsection*{- Glow Patterns }
Fireflies exhibit \textbf{distinct glow patterns}, which vary among species and regions.\\ 
These patterns are often used to attract mates and communicate.\\ 
Each species has a unique flashing pattern, aiding in species identification.

\subsection*{- In Poland }
In Poland, you can observe \textbf{European fireflies} (like \textit{Lampyris noctiluca}) in summer months.

\subsection*{- Spiders catching Fireflies }
Found in \textbf{Orb Weavers}.\\
Predator manipulates the way a fireflies glows to imitate mating calls to lure others into ther den.


\section*{Foxfire (3)}

\subsection*{- Intro }
Foxfire refers to the \textbf{bioluminescence observed in fungi}. Some fungi, such as \textit{Mycena chlorophos},\\ 
glow in the dark, producing a natural light called foxfire.

\subsection*{- Luciferin in Fungi }
This adaptation helps attract insects, which then disperse fungal spores, aiding in reproduction.\\
\textbf{does not need ATP, does use O$_2$}


\subsection*{- Human-modified Luciferin }
Scientists have \textbf{experimented with genetically engineered luciferin} to study its function and build.\\
Luciferin is often widely used by humans, for example as \textbf{Medical markers}.


\section*{Plankton and Algae (3)}

\subsection*{- Blue Tides }
Plankton in coastal regions sometimes form \textbf{"blue tides"},\\ 
a phenomenon where the water emits a glowing light. This is caused by the bioluminescence of \textit{dinoflagellates}, \\
which light up when disturbed by waves or motion.

\subsection*{- "Red" Tides }
We sometimes notice similar phenomenon when the water appears blue,\\
don't assume it's another example of bioluminescence, \textbf{it is not} it's actually caused\\
by excessive amount of \textit{dinoflagellates} in the water, it's \textit{photosynthetic organells} are red and cause this color change.

\subsection*{- Luciferin in Plankton }
In plankton, luciferin and luciferase interact in a reaction that produces light.\\ 
This light serves a specific purpose as a \textbf{defense mechanisms} when they are disturbed.\\
Called \textit{Burglar Allarm} when eaten by fish, glow to lure bigger predator to eat the one that tries to eat them\\
\textbf{no need for ATP, sensitive to PH, needs to be acidic}



\section*{Underwater Monstrosities (11)}

\subsection*{- Jellyfish (2)}
Jellyfish, like the \textbf{moon jellyfish (\textit{Aurelia aurita})}, emit a soft blue light.\\ 
Their bioluminescence helps attract prey and escape predators.\\
One of the most ancient species 550 milion years old\\
Say somehting about the Indo-European \textbf{DEADLY BOX JELLYFISH} the size of a thumb.\\
And something about \textbf{Immortal Jellyfish} that can turn itself into a polyp (to the ground floor) \\
and  theoretically live forever.

\subsection*{- Anglerfish (3)}
The \textbf{anglerfish} uses a unique bioluminescent lure on its head to attract prey in deep-sea habitats.\\
It's actually bacteria in their lure glowing.\\
Their reproduction mechanism is veeery interesitng\\
Male looks for female, fuses into her, their\\
circulatory system merges, male is reduced to the \\
female's testicle

\subsection*{- Octopi (2)}
Some species of \textbf{bioluminescent octopi}, like the deep-sea octopus.\\
They are very intelligent... they can solve puzzles, use tools as coconut shells for shelter\\
They have 9 brains, one main and each for each tentacle\\
They an rapidly change colour on top on their bioluminescence.

\subsection*{- Giant Squid (3)}
The \textbf{giant squid}, known for its massive size (13m) and elusive behavior, \\
uses bioluminescence as a hunting mechanism\\
Largest eyes in the animal kingdom, 30cm in diameter\\
Hunted by \textbf{Sperm Whales} they can dive even up to 3000m\\
Firstly saw alive a few years ago, people lured it using fake glowing jellyfish \\
The Squid wasn't interested in eating the jelly, but was looking around for a "bigger fish"\\

\subsection*{- Kitefin Shark (1)}
The Biggest Bioluminescent animal known

\subsection*{- And many others... }
Other deep-sea creatures, such as various crustaceans, also rely on bioluminescence to navigate,\\
communicate, and escape predators in extreme underwater conditions.


\section*{Bioluminescence and Fluorescence (2)}

\subsection*{- Crucial Difference }
\begin{itemize}
 \item\textbf{Bioluminescence}: Emission of light by a chemical reaction within living organisms.
 \item\textbf{Fluorescence}: Emission of light after absorbing UV light, seen in many animals.
\end{itemize}
  
Bioluminescence is an internal chemical reaction, while fluorescence is caused by \textbf{external UV absorption}.

\subsection*{- Examples: Platypus, Scorpion, Wombat... }
\begin{itemize}
 \item\textbf{Platypus}: Our Real-Life Pokemon exhibits internal fluorescence in some tissues.
 \item\textbf{Scorpions}: Under UV light, their exoskeleton fluoresces.
 \item\textbf{Wombats}: Have internal adaptations that interact with UV light, contributing to survival and communication in their habitats.
\end{itemize}


\section*{Usage in Modern Times (2)}

\subsection*{- Markers for Cancer }
Bioluminescence is widely used in \textbf{medical research}, specifically as a tool to detect \textbf{cancer cells}.\\ 
Scientists attach luciferase genes to cancer cells, allowing imaging techniques to identify tumor locations.

\subsection*{- Glowing Trees? }
\begin{itemize}
 \item What if we \textbf{genetically modified plants}, such as trees, to contain luciferin genes.
 \item \textbf{Energy Efficiency}: We use \textbf{up to 20\%\ of energy consumption} for lighting, 
   we could use glowing trees to contribute to more sustainable urban environments.
\end{itemize}

\hfill
\textbf{Nel Skowronek,Jan Ryszkiewicz}

\end{document}
